\subsection{Product perspective}
\subsection{Product functions}
Here we provide several scenarios to better delineate the purposes for which the app should be designed, the situations the S2B will deal with and more generally to have a better comprehension of the associated environment

	\subsubsection{Scenario 1}
	Mario is the director of his company and he has seen an ad of Travlendar+, so he wants to try it to organize the weekly meetings with his employees. After the installation, he has to register to the app. The first thing he creates is his weekly program. We work from Monday to Friday, from 8 am to 16 pm.\newline
	There is an actual meeting coming up on Friday at 8 pm, so he creates a new	meeting in the app. After setting the time and the day, he invites his employees to join the meeting. He will remain in his office, so he will not need to use	any travel means.\newline
	Giovanni is one of Mario's employee and he is registered to Travlendar+. He receives a notification and accepts the invitation to the meeting. He chooses to reach the location by walk because he will already be nearby.\newline
	Alex is another employee and does not have an account on Travlendar. He receives an email with an invitation link to register to the app. After the registration, the app redirects him to the meeting's invitation and he will proceed by accepting it and choosing to go by car. Then he explores the app and decides to add his weekly program. The app finds out that he will be in the café near the metro at 6:40 pm, so it suggests Alex to take the metro instead of the car. He accepts the suggestion.
	\subsubsection{Scenario 2}
	Paolo, a resident of Bergamo, has recently registered to Travlendar+ and during	the initial setup has specified a flexible launch break from 11:00 am to 13:00 am, with at least 40 minutes of break.\newline
	Tomorrow he is going to have an audition at 12:30 pm, in Monza. He inserts the event in the app and after having specified that the audition will end at about 13:30pm, he looks at the suggestions of the app on the travel means to take: the app suggests him some travel solutions, but he does not specify which he's going to take because he wants to think about it overnight.\newline
	The next morning, the app sends Paolo a reminder with two travel solutions:\\
	\indent- go by car, leaving at about 11.45am, arriving at 12.27 pm;\\
	\indent - take the bus, passing at 10:49 am and arriving at 11:38 am.
	\newline He chooses the second option to avoid being late at the audition.
	\subsubsection{Scenario 3}
	Alex is a professor of Bologna University, he has a short memory and is very badly organized, so he decides to rely on Travlendar+. Alex downloads it on occasion of a work trip. He signs up and decides to insert his credit card data for an eventual purchase from the app.\newline
	He needs to reach the University of Parma to hold a conference. Alex sets up the app to arrive in Parma by train. Travlendar+ asks Alex the kind of event and he specifies it is a formal work meeting. The app asks him which transport means he wants to take in order to get to the university from the railway station. Alex opts to go by bicycle, although the app suggest not to, because of the formal type of meeting.\newline
	The departure day Alex is in a shopping center with his family, he has completely forgotten that he has a train to take, but Travlendar+ solves the problem by notifying him of the appointment. At that point, Alex has no more time and chooses to buy the train tickets using the app. While he is on the train, Travlendar+ suggests him to choose another transport means (instead of a bike) because of the bad weather conditions. Alex accepts the advice and decides to take a bus.
	\subsubsection{Scenario 4}
	Luca is a meteorologist who works for a laboratory in Venice. He knows very well all the climatic problem that humans are creating in their Country. Luca finds Travlendar+ very appropriate to help in solving this problematic. He likes to opt for an Eco-friendly solution by setting up this preference in the app settings. In this way he can avoid, at least in this aspect, further damaging the environment. His favorite functionality is bike sharing because of its innovative localization system and its low environmental impact.
	\subsubsection{Scenario 5}
		Mark, son of Lucas, asks to his father to bring him  to the basketball tournament of Sunday morning.\newline
		Lucas checks the daily schedule for Sunday and he notices he already has an appointment with the hairdresser but, of course, spending time with his son is more important, so he decides to delete the previous event on the agenda and set a new one.\newline
		Mark asks to the father if also his team mate Mike can come.
		Of course Mark and Mike have to bring with them the bags with the jersey and the basketball shoes, so Lucas, creating the event on Travlendar+, after specifying the location of the basketball court, specifies also that he will bring with him baggage and passengers.\newline
		Unfortunately his car is broken, so Lucas use  the app to look for alternative solutions.\newline
		Travlendar+, taking into account the constraints previously settled by the man, suggests to him to use Enjoy or SmartToGo, two well known car sharing companies that will solve his problem.\newline
		Lucas accepts the suggested solution and proceeds with the creation of the event.
	\subsubsection{Scenario 6}
		Mary, John's wife, one week ago, asked to her husband to pick the children up to school on Monday at 13.00 and, because she knows John,  forced him to take note of that with Travlendar+.\newline
		So John planned this event on the app specifying that he will use the car to do it.
		He specified also the location of the school.\newline
		On Monday morning, as usual, Travlndar+ shows to John his daily program reminding him about his children and showing the previous  travel mean planned.\newline
		John, still intentioned to pick the children up with the car, does not modify the plan, closes the app and goes to work in the other side of Milan.\newline
		At 12.00 Travlendar+, according to the GPS position of the man, suggests him to leave in 15 minutes.
		Travlendar+ also suggests to avoid to go through  Viale Gioia because of the traffic and take the SS1.\newline
		Thanks to Travlendar+ John manages to be on time, collect the children and make his wife proud of him. 
\subsection{User characteristics}
	\subsubsection{Actors}
			\begin{itemize}
			\item Person: a person that doesn't have a registered account. The only thing that he/she can do is to proceed with the Sign Up process;
			\item User:a person passed through a successful registration process and now
			able to use all the TravLendar+ services. He/she can login to the system and, after that, use all
			the platform's functionalities.
			\item Credit Institution: the institution that checks the credit-card validity and reports it to the S2B;
			\item Google: the system with whom the S2B retrieves the maps and related information about routes, real-time traffic situations, estimated travel time and weather conditions;
			\item Transport Company System: the system of the affiliated companies with whom the S2B interacts to allow the user to buy the tickets for the associated travel mean;
			\item Shared vehicle Company System: the system with whom the S2B interacts to allow the user to visualize the map of the available vehicles to rent and locate the nearest one. Vehicle Company System provides the GPS locations of the vehicles. To proceed with the renting the user is redirected to the vehicle Company System/App. The latter, to autenticate the user, uses the same credentials provided to the S2B.
			\item Manager: the person able to manage the system. His activities consists in adding or deleting affiliated companies (both  transport companies or shared vehicle companies) and adding new type of tickets according to the services provided by affiliated transport companies.
		\end{itemize}
\subsection{Assumptions, dependencies and constraints}
	\subsubsection{Text Assumptions}
		\begin{itemize}
			\item credit cards do not have an expiration date
		\end{itemize}

	\subsubsection{Domain Assumptions}
		\begin{enumerate}[label={[D\arabic*]}]
			\item \mylabel{D-retrieve-language}{the user's device should allow the app to retrieve the language settings}
			\item \mylabel{D-registration-proceed}{when the registration process begin, the User always inserts his/her credentials}
			\item \mylabel{D-email-always-received}{when the S2B sends an email, it is always received by the receiver}
			\item \mylabel{D-everyone-email}{every Person has an email address}
			\item \mylabel{D-user-remember}{the User shall remember his password}
			\item \mylabel{D-no-user-hacker}{the User knows only his password}
			\item \mylabel{D-working-gps}{the User's device has a working GPS installed, to which the app has access}
			\item \mylabel{D-provide-loc-API}{the affiliated companies provide a localization service APIs}
			\item \mylabel{D-traffic}{Google Maps services take traffic into consideration}
		\end{enumerate}