\subsection{Purpose}
This document is the Requirement Analysis and Specification Document for the Travlendar+ application. Its aim is
to inform about what the application offers, about requirements and goals that the system must present. This document offers also an analysis of the world and of the shared phenomena regarding Travlendar+. RASD contains class diagram to show domain model and other diagrams which illustrate, with more details, transactions of the functionality of the application.
\subsection{Scope}
Travlendar+ is an application that allows people to organize and to track their appointments and meeting by 
registering them into the application. A person becomes a User of Travlendar+ by registering himself to the app. After this phase, users can start to use basic functionalities of the app (e.g. register appointments and organize meetings).\newline 
The app allows users to create appointments, eventually inviting other users of the app, facilitating communication issues.
The goal of Travlendar+ is to organize in the best way all daily commitments of its users considering all the possible problems that can influence travels and trips (e.g. weather conditions, strikes, etc.).\newline
When a User creates an event, he can add, for the travel, eventual passengers or baggage, so that the app can suggest better trip choices. Travlendar+ allows users to visualize their planned schedule too.\newline 
The dynamicity of the software allows users to set some personal preferences, for example, to set a flexible break window time for having free time or lunch, to choose eco-friendly solutions for his trips, to deactivate some transportation means.\newline 
The system interfaces with other firms (e.g. transportation companies, territory maps companies, sharing transportation companies) to offer a more comprehensive user experience. In this way, a User has the possibility to buy tickets and passes for transportation means. Moreover, the system offers a location service for vehicles of affiliated shared transport companies but to proceed with the renting the User is redirected to the company's app.

\subsubsection{Goals}
The system should:
	\begin{enumerate}[label={[G\arabic*]}]
		\item \mylabel{G-dev-language}{allow a Person to use the app in his/her device language, if the language is one of the \defined{supported languages}, English otherwise}
		\item \mylabel{G-become-registered}{allow every Person to have his/her own account, after providing credentials}
		\item \mylabel{G-log-in}{allow a Person to log in to his/her own account, and only in that}
		\item \mylabel{G-use-credit-card}{allow User to use his/her own credit card}
		\item \mylabel{G-pers-preferences}{allow a User to set personal preferences}
		\begin{enumerate}[label=\theenumi\#{\arabic*}]
			\item \mylabel{G-eco-friendly}{specify his/her preference for eco-friendly solution}
			\item \mylabel{G-break-windows}{set break time windows, either flexible or fixed}
			\item \mylabel{G-time-slot}{set time slot in which the use of specific transportation means should be avoided}
			\item \mylabel{G-min-distance}{set a minimum distance below which a specific transportation mean should be avoided}
			\item \mylabel{G-max-distance}{set a maximum distance beyond which a specific transportation mean should be avoided}
			\item \mylabel{G-transp-disabled}{set a specific transportation mean as permanently disabled}
		\end{enumerate}
		\item \mylabel{G-new-appointment}{allow User to create a new appointment, specifying its details}
		\item \mylabel{G-visualize-update}{allow User to visualize his/her appointments, eventually updating its details}
		\item \mylabel{G-choose-sug}{during the specification of details of an appointment, allow User to visualize suggested travel solutions and to choose one of them}
		\begin{enumerate}[label=\theenumi\#{\arabic*}]
			\item \mylabel{G-change-preferred-transp}{during the specification of details of an appointment, before the choice, allow User to set a preferred transportation mean}
		\end{enumerate}
		\item \mylabel{G-invite}{during the specification of details of an appointment, allow User to invite other persons to the appointment}
		\item \mylabel{G-owns-ticket}{during the specification of details of an appointment, if for the selected travel solution a ticket is expected, allow User to specify if he/she owns a ticket, either ordinary or a pass (eventually specifying the deadline)}
		\begin{enumerate}[label=\theenumi\#{\arabic*}]
			\item \mylabel{G-buy-ticket}{if the user does not own a ticket and the transportation company is \defined{affiliated} with Travlendar+, he/she can buy it}
		\end{enumerate}
		\item \mylabel{G-visualize-schedule}{allow User to visualize daily/weekly schedule}
		\item \mylabel{G-delete}{allow User to delete an existing appointment}
		\item \mylabel{G-locate}{allow User to locate nearest vehicle of a vehicle sharing system, if that is the transportation mean of choice of an incoming appointment}
		\item \mylabel{G-redirection}{allow User to being redirected to the company's app to proceed with the renting of the selected shared vehicle}
		\item \mylabel{G-show-suggestions}{show suggestions based on: traffic, weather conditions/forecast, strikes, type of appointment, baggage, passengers}
		\item \mylabel{G-notify}{notify the User about incoming appointments}
		\begin{enumerate}[label=\theenumi\#{\arabic*}]
			\item \mylabel{G-confirmation}{asking for confirmation on the transportation means previously planned}
			\item \mylabel{G-notify-suggestions}{in case no transportation mean was selected, suggesting one}
			\item \mylabel{G-complications}{communicating eventually complications (bad weather, traffic) and suggesting more feasible solutions}
		\end{enumerate}
	\end{enumerate}

\subsection{Definitions, Acronyms, Abbreviations}
	\subsubsection{Definitions}
		\begin{description}
			% TODO Sort alphabetically
			\item[- \textit{welcome page}:] after completing the sign up process,when the user logs in for the first time, he is redirected to this page where sequentially the app asks him to insert credit-card data, driving licence data and set his preferences the user can skip any of this phases and complete them in a second time. After this process, the initial settings configuration is completed
			\item[- \textit{personal preferences}:] with this term we mean that here the user can:\newline
			- specify break time windows; \newline
			- specify the interest or not for eco-friendly solutions;\newline
			- specify constraints, such as avoiding  bike, on the travel means solutions).
			\item[- \textit{supported languages}:]
			\item[- \textit{valid credentials}:]
			\item[- \textit{appointment details}:]
			\item[- \textit{incoming appointment}:]
			\item[- \textit{affiliated company}:]
		\end{description}
	\subsubsection{Acronyms}
		\begin{description}
		\item[- \textit{S2B}:] System to Be;
		\item[- \textit{API}:] Application Programming Interface.
	\end{description}
	\subsubsection{Abbreviations}
\subsection{Revision history}
\subsection{Document Structure}