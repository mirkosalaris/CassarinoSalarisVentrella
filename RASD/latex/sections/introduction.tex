\subsection{Purpose}
This document is the Requirement Analysis and Specification Document for the Travlendar+ application. Its aim is
to inform about what the application offers, about requirements and goals that the system must present. This document offers also an analysis of the world and of the shared phenomena regarding Travlendar+. RASD contains class diagram to show domain model and other diagrams which illustrate, with more details, transactions of the functionalities of the application.
\subsection{Scope}
Travlendar+ is an application that allows people to organize and to track their appointments and meeting by 
registering them into the application. A person becomes a User of Travlendar+ by registering himself to the app. After this phase, users can start to use basic functionalities of the app (e.g. register appointments and organize meetings).\newline 
The app allows users to create appointments, eventually inviting other users of the app, facilitating communication issues.
The goal of Travlendar+ is to organize in the best way all daily commitments of its users considering all the possible problems that can influence travels and trips (e.g. weather conditions, strikes, etc.).\newline
When a User creates an event, he can add, for the travel, eventual passengers or baggage, so that the app can suggest better trip choices. Travlendar+ allows users to visualize their planned schedule too.\newline 
The dynamicity of the software allows users to set some personal preferences, for example, to set a flexible break window time for having free time or lunch, to choose eco-friendly solutions for his trips, to deactivate some transportation means.\newline 
The system interfaces with other firms (e.g. transportation companies, territory maps companies, sharing transportation companies) to offer a more comprehensive user experience. In this way, a User has the possibility to buy tickets and passes for transportation means. Moreover, the system offers a localization service for vehicles of affiliated shared transport companies but to proceed with the renting the User is redirected to the company's app.

\subsubsection{Goals}
	\begin{enumerate}[label={[G\arabic*]}]
		\item \mylabel{G-become-registered}{a Person should be able to have his/her own Travlendar+ agenda}
		\item \mylabel{G-pers-preferences}{a User should be able to customize the offered service}
		\begin{enumerate}[label=\theenumi\#{\arabic*}]
			\item \mylabel{G-eco-friendly}{specify his/her preference for eco-friendly solution}
			\item \mylabel{G-break-windows}{define break time windows, either flexible or fixed}
			\item \mylabel{G-time-slot}{define time slot in which the use of specific transportation means should be avoided}
			\item \mylabel{G-min-distance}{define a minimum distance below which a specific transportation mean should be avoided}
			\item \mylabel{G-max-distance}{define a maximum distance beyond which a specific transportation mean should be \newline avoided}
			\item \mylabel{G-transp-disabled}{disable permanently specific transportation means}
		\end{enumerate}
		\item \mylabel{G-all-appointment}{a User should be able to take note of all his/her appointments and their details}
		\item \mylabel{G-manage-agenda}{a User should be able to manage his/her appointments}
		\item \mylabel{G-choose-sug}{for each appointment, the User should be assisted in the choice of the travel solution}
		\begin{enumerate}[label=\theenumi\#{\arabic*}]
			\item \mylabel{G-show-suggestions}{travel solution suggestions must take into account traffic, weather conditions/forecast, strikes, type of appointment, baggage, passengers}			
		\end{enumerate}
		\item \mylabel{G-invite}{a User should be able to invite other persons to his/her appointment}
		\item \mylabel{G-owns-ticket}{a User is assisted in the purchase of a ticket when it is required}
		\item \mylabel{G-locate}{a User should be able to locate the nearest vehicle of a vehicle sharing system, if that is the transportation mean of choice of an incoming appointment}
		\item \mylabel{G-redirection}{a User should be able to rent a shared vehicle, if that is the transportation mean of choice of an incoming appointment}
		\item \mylabel{G-notify}{a User should always be aware of the incoming appointments and how to reach them}
		\begin{enumerate}[label=\theenumi\#{\arabic*}]
			\item \mylabel{G-complications}{the User should be aware of eventual complications (bad weather, traffic, strikes)}
		\end{enumerate}
	\end{enumerate}

\subsection{Definitions, Acronyms, Abbreviations}
	\subsubsection{Definitions}
		\label{definitions}
		Here we provide a list of definitions of words and expression used in the documents. Every time such words or expressions will be used they will be preceded by the symbol "($\uparrow$)" that will be a link to this section.
		\begin{description}
			\item[- \textit{affiliated company}:] a company (transportation company or vehicle sharing company) that has deals with the S2B.
			\item[- \textit{appointment details}:] time, date, type of appointment, location, number of passengers and presence of baggage.
			\item[- \textit{break window}:] it is a time slot in which the User specifies he/she wants to be absolutely free. In this time slot he does not want to have any appointment nor to be travelling. This is a fixed break window. A flexible break window is a time slot in which the User specifies he/she wants to have some free time, but not the full window free.
			\item[- \textit{incoming appointment}:] the next scheduled appointment for which the S2B send a reminder to the user. The reminder is sent  a certain amount of time before the appointment starts, to allow the user to get on time in the location.
			\item[- \textit{personal preferences}:] with this term we mean that here the user can:\newline
			- specify break time windows; \newline
			- specify the interest or not for eco-friendly solutions;\newline
			- specify constraints, such as avoiding  bike, on the travel means solutions.
			\item[- \textit{supported languages}:] the set of languages that the S2B will be able to use to communicate with its users. English and Italian are included.
			\item[- \textit{valid credentials}:] Name, surname, personal email address and password.
			\begin{itemize}
				\item[Name:] it should be non empty and it should contain only alphabetical characters
				\item[Surname:] it should be non empty and it should contain only alphabetical characters
				\item[Email:] It should be a valid email address, with an alphanumerical string followed by a '@', followed by an alphanumerical string, a dot, and a domain name
				\item[Password:] It should be a string with at least 8 characters
			\end{itemize}
			\item[- \textit{welcome page}:] the page where the user is redirected after completing the sign-up process and logging in for the first time. In this page, the app sequentially asks the user to insert credit-card data and set his preferences. The user can skip any of this phases and complete them at a later time. After this process, the initial settings configuration is completed.
		\end{description}
	\subsubsection{Acronyms}
		\begin{description}
		\item[- \textit{S2B}:] System to Be;
		\item[- \textit{API}:] Application Programming Interface.
	\end{description}
	\subsubsection{Abbreviations}
		\begin{description}
			\item[- \textit{[Gn]}:] nth goal. Apart when it is actually defined, it is always a reference to the definition of the goal
			\item[- \textit{[Dn]}:] nth domain assumption. Apart when it is actually defined, it is always a reference to the definition of the domain assumption
			\item[- \textit{[Rn]}:] nth requirement
		\end{description}
\subsection{Revision history}
\subsection{Document Structure}
After purpose and scope, used to briefly introduce the topic, we delineated the goals that the S2B should achieve coupled with a list of useful definitions and acronyms.
Subsequently, the text proceeds with an analysis of the functions that the app should provide. The analysis starts with a general exposition of the scenarios and becomes gradually more detailed passing through the analysis of the actors that will interact with the S2B and the statement of the domain assumptions.
After that, the specific requirements are exposed focusing firstly on the external interfaces and then providing the models used to highlight the relations between actors and S2B and describe the internal structure of the latter.
After that, Functional and non-Functional requirements are sequentially discussed.
Before ending with the effort spent and the references we provide a formal analysis performed with alloy.