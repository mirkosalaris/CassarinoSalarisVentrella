\subsection{Purpose}
This document is the Requirement Analysis and Specification Document for the Travlendar+ application. Its aim is
to inform about what the application offers, about requirements and goals that the system must present. This document offers also an analysis of the world and of the shared phenomena regarding Travlendar+. RASD contains class diagram to show domain model and other diagrams which illustrate, with more details, transactions of the functionality of the application.
\subsection{Scope}
Travlendar+ is an application that allows people to organize and to track their appointments and meeting by 
registering them into the application. A person becomes a User of Travlendar+ by registering himself to the app. After this phase, users can start to use basic functionalities of the app (e.g. register appointments and organize meetings).\newline 
The app allows users to create appointments, eventually inviting other users of the app, facilitating communication issues.
The goal of Travlendar+ is to organize in the best way all daily commitments of its users considering all the possible problems that can influence travels and trips (e.g. weather conditions, strikes, etc.).\newline
When a User creates an event, he can add, for the travel, eventual passengers or baggage, so that the app can suggest better trip choices. Travlendar+ allows users to visualize their planned schedule too.\newline 
The dynamicity of the software allows users to set some personal preferences, for example, to set a flexible break window time for having free time or lunch, to choose eco-friendly solutions for his trips, to deactivate some transportation means.\newline 
The system interfaces with other firms (e.g. transportation companies, territory maps companies, sharing transportation companies) to offer a more comprehensive user experience. In this way, a User has the possibility to buy tickets and passes for transportation means. Moreover, the system offers a location service for vehicles of affiliated shared transport companies but to proceed with the renting the User is redirected to the company's app.

\subsubsection{Goals}
	\begin{enumerate}[label={[G\arabic*]}]
		\item \mylabel{G-become-registered}{a Person should be able to have his/her own Travlendar+ agenda}
		\item \mylabel{G-pers-preferences}{a User should be able to customize the offered service}
		\begin{enumerate}[label=\theenumi\#{\arabic*}]
			\item \mylabel{G-eco-friendly}{specify his/her preference for eco-friendly solution}
			\item \mylabel{G-break-windows}{define break time windows, either flexible or fixed}
			\item \mylabel{G-time-slot}{define time slot in which the use of specific transportation means should be avoided}
			\item \mylabel{G-min-distance}{define a minimum distance below which a specific transportation mean should be avoided}
			\item \mylabel{G-max-distance}{define a maximum distance beyond which a specific transportation mean should be avoided}
			\item \mylabel{G-transp-disabled}{disable permanently specific transportation means}
		\end{enumerate}
		\item \mylabel{G-all-appointment}{a User should be able to take note of all his/her appointments and their details}
		\item \mylabel{G-manage-agenda}{a User should be able to manage his/her appointments}
		\item \mylabel{G-choose-sug}{for each appointment, the User should be assisted in the choice of the travel solution}
		\begin{enumerate}[label=\theenumi\#{\arabic*}]
			\item \mylabel{G-show-suggestions}{travel solution suggestions must take into account traffic, weather conditions/forecast, strikes, type of appointment, baggage, passengers}			
		\end{enumerate}
		\item \mylabel{G-invite}{a User should be able to invite other persons to his/her appointment}
		\item \mylabel{G-owns-ticket}{a User is assisted in the purchase of a ticket when it is required}
		\item \mylabel{G-locate}{a User should be able to locate nearest vehicle of a vehicle sharing system, if that is the transportation mean of choice of an incoming appointment}
		\item \mylabel{G-redirection}{a User should be able to rent a shared vehicle, if that is the transportation mean of choice of an incoming appointment}
		\item \mylabel{G-notify}{a User should always be aware of the incoming appointments and how to reach them}
		\begin{enumerate}[label=\theenumi\#{\arabic*}]
			\item \mylabel{G-complications}{the User should be aware of eventual complications (bad weather, traffic, strikes)}
		\end{enumerate}
	\end{enumerate}

\subsection{Definitions, Acronyms, Abbreviations}
	\subsubsection{Definitions}
		\begin{description}
			% TODO Sort alphabetically
			\item[- \textit{welcome page}:] the page where the user is redirected after completing the sign up process and logging in for the first time. In this page the app sequentially  asks to the user to insert credit-card data and set his preferences. The user can skip any of this phases and complete them in a second time. After this process, the initial settings configuration is completed.
			\item[- \textit{personal preferences}:] with this term we mean that here the user can:\newline
			- specify break time windows; \newline
			- specify the interest or not for eco-friendly solutions;\newline
			- specify constraints, such as avoiding  bike, on the travel means solutions).
			\item[- \textit{supported languages}:] the set of languages that the S2B will be able to use to comunicate with the user. English and Italian are included.
			\item[- \textit{valid credentials}: Name, surname and personal email address]
			\item[- \textit{appointment details}:] time, date, type of appointment, location, number of passengers and precence of baggage.
			\item[- \textit{incoming appointment}:] the next scheduled appointment for which the S2B send a remind to the user. The remind is sent  a certain amount of time before the appointment starts, to allow the user to get on time in the location.
			\item[- \textit{affiliated company}:] a company (transport company or vehicle sharing company) that has deals with the S2B.
		\end{description}
	\subsubsection{Acronyms}
		\begin{description}
		\item[- \textit{S2B}:] System to Be;
		\item[- \textit{API}:] Application Programming Interface.
	\end{description}
	\subsubsection{Abbreviations}
		\begin{description}
			\item[- \textit{[Gn]}:] n-th goal. Apart when it is actually defined, it is always a reference to the definition of the goal
			\item[- \textit{[Dn]}:] n-th domain assumption. Apart when it is actually defined, it is always a reference to the definition of the domain assumption
			\item[- \textit{[Rn]}:] n-th requirement
		\end{description}
\subsection{Revision history}
\subsection{Document Structure}