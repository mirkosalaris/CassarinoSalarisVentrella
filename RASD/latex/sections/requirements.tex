\subsection{External Interface Requirements}
	\subsubsection{User Interfaces}
		Here we provide some basic mockups to show how the interface should appear to the user:\newline
		
		\begin{figure}[H]	
			\centerline{\includegraphics[scale=0.4]{Images/Login}}
			\caption{Login}
		\end{figure}	
		\begin{figure}[H]		
			\centerline{\includegraphics[scale=0.4]{Images/suggestedsolution}}
			\caption{Select solution}
		\end{figure}	
		\begin{figure}[H]
			\centerline{\includegraphics[scale=0.5]{Images/schedule}}
			\caption{Visualize schedule}
		\end{figure}
	
	\subsubsection{Hardware Interfaces}
	The main hardware interface of the system consists in the access to the GPS data
	in the mobile application. The application also requires Internet connectivity
	and internal storage access.
	
	\subsubsection{Software Interfaces}
	The mobile application must support Android,iOS and the remaining main OSs (further details are discussed in paragraph 3.6.5 Portability). The web application
	works on any web server that supports Java. The back-end stores its data in a
	DBMS and can run on every platform that supports the JVM.
	\subsubsection{Communication Interfaces}
\subsection{UML modeling}
	\subsubsection{Use Case diagram}
		\smallskip
		\begin{figure}[H]	
			\centerline{\includegraphics[width=\paperwidth-1]{Images/UseCaseDiagram1}}
			\caption{User Use Case diagram}
		\end{figure}

	\begin{figure}[H]	
	\centerline{\includegraphics[width=\paperwidth-1]{Images/UseCaseDiagram2}}
	\caption{ specify details }
\end{figure}

	\begin{figure}[H]	
	\centerline{\includegraphics[scale= 0.7]{Images/UseCaseDiagram3}}
	\caption{Manager Use Case diagram}
\end{figure}


		\renewcommand{\arraystretch}{1.6} % increase line height
		
		\medskip %leave a bit of space before the next 'section'
		\vbox{ % to avoid page breaking
			\noindent
			\textbf{User creates appointment}
			\medskip\\
			\begin{tabu} to \textwidth {| X[\fcWidth,r,p] | X[1-\fcWidth,l,p] |}
				\hline\textbf{Use case:} & User creates an appointment
				\\
				\hline\textbf{Actors:} & User
				\\
				\hline\textbf{Entry condition:} & The user must be logged
				\\
				\hline\textbf{Flow of events:} & The user creates an event (a meeting, appointment or generic event) giving it a name;\newline
				User specifies the location of the event;\newline
				User specifies details such as passengers or baggage;\newline
				User selects a travel mean taking to account app’s suggestion;\newline
				The app takes note of the settings and send a confirmation;\newline
				The app redirects the user to the main page.
				\\
				\hline\textbf{Secondary flows:} & User does not specify a travel mean and let it blank;\newline
				The app takes anyway note of the setting and alert the user of the missing information; \newline
				The app redirects the user to the main page.
				\\
				\hline\textbf{Exceptions:} & Warnings messages are created in the following cases:
				\begin{itemize}
					\item User creates an event that overlaps another event;
					\item User creates an event with a location that is unreachable in the allocated time;
					\item User creates an event that violates the set constraints about the break windows.
				\end{itemize}
				\\
				\hline\textbf{Post conditions:} & The user is successfully redirected to the main page.
				\\
				\hline
			\end{tabu}
			\bigskip %leave a bit of space after the table
		}
		
		\vbox{ % to avoid page breaking
			\noindent
			\textbf{Sign Up}
			\medskip\\
			\begin{tabu} to \textwidth {| X[\fcWidth,r,p] | X[1-\fcWidth,l,p] |}
				\hline\textbf{Use case:} & Sign Up
				\\
				\hline\textbf{Actors:} & Person
				\\
				\hline\textbf{Entry condition:} &user must be not already logged;
				\\
				\hline\textbf{Flow of events:} & the person inserts full name and email contact;\newline
				the app send an email with the confirmation link;\newline
				the person give the confirmation trough the link on the mail;\newline
				the app give the welcome to the new user.
				
				\\
				\hline\textbf{Secondary flows:} & none.
				\\
				\hline\textbf{Exceptions:} &the person inserts a wrong email contact;\newline
				the sign up cannot proceed.
				
				\\
				\hline\textbf{Post conditions:} & the person is successfully signed up and become an actual logged user.
				\\
				\hline
			\end{tabu}
			\bigskip %leave a bit of space after the table
		}
		
		\vbox{ % to avoid page breaking
			\noindent
			\textbf{Initial settings configuration}
			\medskip\\
			\begin{tabu} to \textwidth {| X[\fcWidth,r,p] | X[1-\fcWidth,l,p] |}
				\hline\textbf{Use case:} & Initial settings configuration
				\\
				\hline\textbf{Actors:} & User
				\\
				\hline\textbf{Entry condition:} & the User just has just completed the sign up process;\newline
				user must be logged.
				
				\\
				\hline\textbf{Flow of events:} & User insert sequentially the following  information:
				\begin{itemize}
				\item 	Credit card;
				\item 	Driving licence;
				\item 	Break time windows;
				\item	Interest for Eco-friendly solutions on the travel means.
				\item	Constraints on travel means.
				\end{itemize}
				The app, for each step,  check the info and send a confirmation;
				The app redirects the user to the main page.
				
				\\
				\hline\textbf{Secondary flows:} & User skips to specify one or more information that could be specified later in the settings.\newline
				The app notifies the user about the missing information and redirects anyway the user to the main page.
				
				\\
				\hline\textbf{Exceptions:} & user inserts inconsistent information (incorrect credid-card/licence information, break time shorter than 30 minutes);\newline
				The app allerts the user and asks him to insert again the info.
				\\
				\hline\textbf{Post conditions:} & the set configurations are successfully saved and the user is redirected to the main page.
				\\
				\hline
			\end{tabu}
			\bigskip %leave a bit of space after the table
		}
		
		\vbox{ % to avoid page breaking
			\noindent
			\textbf{User visualizes the appointment}
			\medskip\\
			\begin{tabu} to \textwidth {| X[\fcWidth,r,p] | X[1-\fcWidth,l,p] |}
				\hline\textbf{Use case:} & User visualizes the appointment.
				\\
				\hline\textbf{Actors:} & User
				\\
				\hline\textbf{Entry condition:} & user must be logged;\newline
				the user must have selected the appointment from the schedule;
				
				\\
				\hline\textbf{Description:} & user visualize an appointment and eventually:
				\begin{itemize}
					\item If not specified yet, set a travel mean;
					\item Modify the previous travel mean or other detail such as location, baggage or passengers;
					\item Buy a ticket for the travel mean;
					\item Rent a shared vehicle if is the proper time to do that and locate the nearest one;
					\item Delete the appointment. 
				\end{itemize}\\
				\hline
			\end{tabu}\\
			\bigskip %leave a bit of space after the table
		}
		
		\vbox{ % to avoid page breaking
			\noindent
			\textbf{User buy travel ticket}
			\medskip\\
			\begin{tabu} to \textwidth {| X[\fcWidth,r,p] | X[1-\fcWidth,l,p] |}
				\hline\textbf{Use case:} & User buy travel ticket
				\\
				\hline\textbf{Actors:} & User
				\\
				\hline\textbf{Entry condition:} & user must be logged; \newline
				user must have added a payment card .
				
				\\
				\hline\textbf{Flow of events:} &User select an event;\newline
				User,  through the app, searches for tickets for the specified travel mean; \newline
				User select the ticket’s option and picks one;\newline
				User proceeds with the payment;\newline
				The payment operation ends successfully;\newline
				The app send a confirmation and redirects the user to the main page;
				
				\\
				\hline\textbf{Secondary flows:} & none
				\\
				\hline\textbf{Exceptions:} & The payment is rejected (not enough credit, expired card, ..);\newline
				The app notifies the user; \newline
				The app redirect the user to the home page;
				
				\\
				\hline\textbf{Post conditions:} & User successfully books the tickets and is redirected to the main page.
				\\
				\hline
			\end{tabu}
		}
	\subsubsection{Class diagram}
		% don't really know what happens here, but it is fine graphically
		% ((I'm referring to '\paperwidth-1')
		\begin{figure}[H]
			\centerline{\includegraphics[width=\paperwidth-1]{Images/ClassDiagram}}
			\caption{General class diagram}
		\end{figure}
	\subsubsection{Activity diagrams}
		\begin{figure}[H]
			\centerline{\includegraphics[height=0.75\paperheight]{Images/RegistrationDiagramAD}}
			\caption{Registration activity diagram}
		\end{figure}
		\begin{figure}[H]
			\centerline{\includegraphics[width=\paperwidth-1]{Images/CreationAppointmentAD}}
			\caption{Creation appointment diagram}
		\end{figure}
		\begin{figure}[H]
			\centerline{\includegraphics[height=0.35\paperheight]{Images/SolutionSelectionAD}}
			\caption{Solution selection sub-activity}
		\end{figure}
		\begin{figure}[H]
			\centerline{\includegraphics[height=0.35\paperheight]{Images/TicketHandlingAD}}
			\caption{Ticket handling sub-activity}
		\end{figure}

		

%		\begin{figure}[H]
%			\centerline{\includegraphics[width=\paperwidth-1]{Images/AppointmentCreationDiagram}}
%			\caption{Appointment creation diagram}
%		\end{figure}

\subsubsection{Sequence Diagram}
We chose to omit the Sequence Diagram considering that the Activity diagram in this phase will fit better.
\subsection{Functional Requirements}
	\ref{G_dev_language} 
	\begin{enumerate}[label={[R\arabic*]}]
		\item the app must retrieve the language settings from the device
		\item the app should be able to display its content in all the \defined{supported languages}
	\end{enumerate}
	\begin{itemize}
		\item[] \ref{D_retrieve_language}
	\end{itemize}

	\noindent\ref{G_become_registered}
	\begin{enumerate}[resume, label={[R\arabic*]}]
	
		\item \label{R_begin_registration} the system must provide a way to begin the registration process
		\item \label{R_send_email} after the insertion of the credentials and their \defined{validation}, the system has to send to the provided address an email with an activation link
		\item \label{R_unique_email} the registration fails if the inserted email is already associated to an account
		\item \label{R_user_after_activation} when the Person confirms through the activation link, he/she becomes a User
		\item \label{R_restart} in the case of non \defined{valid credentials}, the system must reject them and restart the registration process
	\end{enumerate}
	\begin{itemize}
		\item[] \ref{D_email_always_received}
	\end{itemize}
	
	\noindent\ref{G_new_appointment}
	\begin{enumerate}[resume, label={[R\arabic*]}]
	\item \label{R_new_appointment} the system must provide a way to start the creation of a new appointment
	\item \label{R_appointment_details} during the process the user shall insert the \defined{appointment details}	
	\end{enumerate}
	
	\noindent\ref{G_visualize&update}
	\begin{enumerate}[resume, label={[R\arabic*]}]
		\item \label{R_visualize} the system provides a way to visualize created appointments
		\item \label{R_edit_details} after visualizing an appointment, a user can choose to edit its details
	\end{enumerate}
	
	\noindent\ref{G_choose_sug} and \ref{G_change_preferred_transp}
	\begin{enumerate}[resume, label={[R\arabic*]}]
	\item \label{R_solutions} when time and location of the current appointment are set, the S2B produces a list of travel solutions with associated suggestions
	\item \label{R_choose_solution} the S2B provide the user the possibility to choose one of the travel solutions or leave the travel plan unspecified
	\item \label{R_select_preferred} the S2B also provide the possibility to choose a preferred transportation mean
	\item \label{R_recompute_solutions} when a new preferred transportation mean is selected the S2B has to recompute the list of solutions according to the new preference
	\end{enumerate}
	
	\noindent\ref{G_invite}
	\begin{enumerate}[resume, label={[R\arabic*]}]
	\item \label{R_can_invite} when time and location of the current appointment are set, the S2B offers the possibility to invite other users or persons, through their emails.
	\item \label{R_invite_send_email} when a User or a Person is invited, the S2B will inform him/her sending an email
	\end{enumerate}
	\begin{itemize}
		\item[] \ref{D_email_always_received}
	\end{itemize}
	
	\noindent\ref{G_owns_ticket}
	\begin{enumerate}[resume, label={[R\arabic*]}]
	\item when the user selects a travel solution for which a ticket is exepected, the S2B asks the User specify if he/she owns a ticket (either ordinary or a pass, in which case the deadline has to be inserted)
	\item if the user has selected a travel solution for which a ticket is expected and the User said to not own a ticket, the S2B asks him/her to buy a ticket (options available only for transportation means of \defined{affiliated companies})
	\end{enumerate}
	
	\noindent\ref{G_visualize_schedule}
	\begin{enumerate}[resume, label={[R\arabic*]}]
	\item \label{R_schedule_view} existing appointments can be viewed together as a schedule view
	\item \label{R_daily_weekly} the schedule can be daily or weekly
	\end{enumerate}
	
	\noindent\ref{G_delete}
	\begin{enumerate}[resume, label={[R\arabic*]}]
		\item \label{R_can_select} in the schedule view the User can select one or more appointments
		\item \label{R_delete_selected} in the schedule view, selected appointments can be deleted
	\end{enumerate}
	
	\noindent\ref{G_locate}
	\begin{enumerate}[resume, label={[R\arabic*]}]
	\item \label{R_provide_loc_service} when an user visualize an incoming appointment for which a vehicle sharing system has to be used, the S2B provides a localization service
	\end{enumerate}
	\begin{itemize}
		\item[] \ref{D_working_gps}
		\item[] \ref{D_provide_loc_API}
	\end{itemize}
	
	\noindent\ref{G_redirection}
	\begin{enumerate}[resume, label={[R\arabic*]}]
		\item \label{R_rent_vehicle} the S2B offers the possibility to rent a vehicle
		\item \label{R_redirects} when the user selects a vehicle to rent, the app redirects him/her to the right company's app or site
	\end{enumerate}
	
	\noindent\ref{G_show_suggestions}
	\begin{enumerate}[resume, label={[R\arabic*]}]
	\item \label{R_bad_weather} if weather forecast are bad: foot, bicycle motorbike are discouraged
	\item \label{R_strikes} if strikes have been announced, public transport is discouraged
	\item \label{R_baggage_passengers} in case of baggage or passengers a car is recommended
	\item \label{R_work} in case of a work appointment, bicycle is discouraged
	\end{enumerate}
	\begin{itemize}
		\item[] \ref{D_traffic}
	\end{itemize}
	
	
	\noindent\ref{G_notify}
	\begin{enumerate}[resume, label={[R\arabic*]}]
	\item \label{R_incoming_notified} when an appointment becomes incoming, a notification is sent to the user
	\item \label{R_notChoosen_suggests} if a travel plan has not already been choosen, the notification suggests one
	\item \label{R_choosen_update} if a travel plan has already been choosen but at the moment better travel plans are available (according to new weather forecast and news on traffic and strikes), the notification suggests a new feasible solution
	\end{enumerate}
	
\subsection{Performance Requirements}
The system has to be able to respond to a possibly great number of simultaneous requests and
more generally to a great number of request throughout the day.
The S2B, at least for the start, will only be available for the Lombardy region. Based on demographic analysis (number of inhabitants, number of people under the age of 60, number of smartphones sold over the past 2 years), it was decided to design the S2B to support 100,000 users simultaneously, but scalability needs to be guaranteed.
\subsection{Design Constraints}
	\subsubsection{Standard compliance}
	To ensure interoperability the S2B will follow the W3C web standard and will be as adherent as possible to  coding practices in relation to the use of HTML/XHTM, CSS and Java programming language
	Moreover the use of private libraries will be avoided.
	\subsubsection{Hardware limitations}
		\begin{itemize}
		\item Mobile App: \newline
		* 3G connection\newline
		* GPS\newline
		* Space for app package
		\item Web App:\newline
		* Modern browser with AJAX support
		\end{itemize}
	
	\subsubsection{Any other constraint}
	Regulatory policies
	The system will ask for users' payment informations and obviously, in addition to store them
	safely, will use them only for fees and rides payments.
	Moreover, the system will have to ask for users' permission in order to retrieve and use their
	positions.
	Email addresses won't be used for commercial uses.
\subsection{Software System Attributes} 
	\subsubsection{Reliability}
	The system must guarantee a 24/7 service. Very small deviations from this requirement will be
	obviously acceptable.
	\subsubsection{Availability}
	The S2B must guarantees a 3-nines availability (99.9 percent) with a downtime not greater than 8 hours per year.
	\subsubsection{Security}
User credentials and payment information will be stored. Data confidentiality is a primary concern.
In addition,when the user wants to buy tickets or rent a shared vehicle, the stored information must be sent to affiliated  transport Company systems or shared vehicle Company  systems . To ensure the security and the confidentiality of this information, the S2B must be able to adopt  access managment protocols and comunication protocols able to prevent not granted access and/or Sniffing/Spoofing activities performed by third.
	
	\subsubsection{Maintainability}
	The S2B must be designed in the way to easly correct defects or their cause,
	repair or replace faulty or worn-out components without having to replace still working parts,
	prevent unexpected working condition,
	maximize its useful life,
	maximize efficiency, reliability, and safety,
	meet new requirements,
	make future maintenance and
	cope with a changed environment.
	
	\subsubsection{Portability}
	The S2B must be able to run in all main mobile OS(Android, iOS, Nokia OS, blackBerry OS, windows-phone OS) and being supported by all the main Web Browser(Google Crome, Safari,FireFox, Microsoft Edge, Internet Explorer, ...)\newline
	
