\documentclass[a4paper]{article}
\usepackage{graphicx} % to include graphicx elements
\usepackage{tabu} % to use tables
\usepackage{geometry} % to define page layout
\usepackage[pdftex, colorlinks]{hyperref} % to use href

\usepackage[dvipsnames]{xcolor}
\usepackage{listings}
\usepackage{enumitem}
\usepackage{float}
\usepackage{alloy-style}
\usepackage{enumitem}

\colorlet{mylinkcolor}{Orange}
\colorlet{mycitecolor}{Orange}
\colorlet{myurlcolor}{Aquamarine}
\hypersetup{
	linkcolor  = mylinkcolor!80!black,
	citecolor  = mycitecolor!80!black,
	urlcolor   = myurlcolor!80!black,
	colorlinks = true,
}

% used to identify words defined in definitions section
\newcommand{\defined}[1]{{\hyperref[definitions]{($\uparrow$)}\textit{#1}}}

% this is to create labels like \mylabel{arg1}{arg2}. The first one is the part that you call with \ref{arg1} (or with \myref), the second part will be just copy-pasted after the ref link
\makeatletter
\newcommand{\mylabel}[2]{{#2} \label{#1}%
	{\protected@write \@auxout {}{\string \newlabel {#1.text}{{~#2}{\thepage}{}{}{}} } \hypertarget{#1}{}}}
\makeatother

% to be used to actually exploit \mylabel
\newcommand{\myref}[1]{\ref{#1}\ref{#1.text}}

\newcommand{\goal}[1]{\bigskip
	\noindent\textbf{\myref{#1}}}

% set margin of enumerate sections
\setlist[enumerate,1]{leftmargin=1.2cm}

%----- Define page layout -----
 \geometry{
 left=25mm,
 right=25mm,
 top=30mm,
 bottom=25mm,
 headsep=7mm}

%----- Constant definitions -----
\def \fcWidth {0.2} % first column width