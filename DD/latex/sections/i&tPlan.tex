With the goal of identifying a proper developing plan, for each feature we have analyzed:
\begin{itemize}
	\item the importance for Users: how much the feature is perceived as fundamental, from the User point of view. Core features have been labeled with a high or medium-high importance. They have to be implemented completely during the first releases. Features with importance low or medium-low are the ones that could be implemented in successive releases as additional features that add value to Travlendar+ but are not part of the core functionalities.
	\item difficulty of implementation of the Back End: it defines an estimate of the effort that will be spent in the implementation of the communication issues, the logic and the data models on the server required by the feature.
	\item difficulty of implementation of the Front End: this indicates an estimate of the effort required to implement in the App \textit{(Mobile App or Web App)} the graphic and the user interactions
\end{itemize}

\noindent
The following table summarizes the result of our analysis.

\begin{center}
	\renewcommand{\arraystretch}{1.5}
	\rowcolors{1}{gray!20}{white}
	\begin{tabular}{|p{0.4\textwidth} l l l|}
		\hline
		\rowcolor{gray!45}
		\textbf{Feature} & \textbf{Importance for User} & \textbf{Back End} & \textbf{Front End}\\

		sign up 					& medium-low 	& low 			& low \\
		appointment creation and management	& high 	& medium-low	& medium-low \\
		schedule visualization		& high 			& low			& medium-high \\
		settings 					& medium-low 	& low			& medium-low \\
		solutions calculation 		& high 			& medium-high	& medium-high \\
		notification system 		& medium-high 	& medium-high	& low \\
		weather \& traffic updates 	& medium-high 	& high			& low \\
		tickets purchase and management	& low 		& medium-high	& medium-high \\
		localization sharing vehicle& medium-low 	& medium-low	& medium-low \\
		rent a sharing vehicle		& low 			& low			& low \\
		invitations					& medium-low	& medium-high	& medium-low \\
		Incoming Appointments		& medium-high	& medium-low	& low \\
		\hline
	\end{tabular}
\end{center}

\bigskip
\noindent
Verification and Validation process will begin as soon as the system starts to be implemented. For each component (and for each subcomponent belonging to it) there are two phases: the first one is an analysis concerning in a manual inspection by team members, the second one is an automatized test using tools support. We chose to make a manual inspection of the product because, in this way, we have a major formal level compared to a Walkthroughs analysis; in fact, inspection is an expensive technique, but it guarantees a low level of errors in the final product.\newline
Below there is a list ordered (time increasing) by system features and not by components because, for the integration process, it has been chosen to adopt Thread strategy. In this way we can release to the customer more than one version of the product, from the version that contains core functionalities (all of high importance for the user) to the complete version, adding step by step new features.
Thread strategy allows implementing also a part of a specific component. For instance, considering Solution Creator component, at the beginning it can be implemented, analyzed and tested without taking into account the possibility to specify settings (medium-low priority for the user). Setting features can be added to the system at a later time.\bigskip \newline 

\noindent
Components will be implemented, integrated and tested in the following order:

\begin{description}[before={\renewcommand{\makelabel}[1]{ \textbf{\textit{##1}}:}}]
		
	\item[1. Appointment creation and management] to satisfy this functionality, part of the Appointment Manager is implemented, analyzed and tested. In particular, only two of its subcomponents are implemented, Appointment Editor and Appointment Handler.
	\item[2. Solution calculation] to satisfy this functionality, part of the Solution Calculator is implemented, analyzed and tested. Solution Calculator, at this point, interfaces with the implemented part of Appointment Manager and with Google API. It doesn't interface either with Account Manager (solutions are processed without taking into account user's settings) or with Weather and Traffic modules (solutions are calculated without considering bad weather and traffic).
	\item[3. Sign up] to satisfy this functionality, part of Account Manager is implemented, analyzed and tested. In particular, the part that allows Users to sign up into the system in order to be recognized during future accesses.
	\item[4. Schedule visualization] to satisfy this functionality, the implementation, analysis, and test of Appointment Manager continue. In it, there is not any additional component but, in this phase, we handle the graphics interface concerning the schedule visualization \textit{(medium-high in front-end)}.\newline
	It is possible to make a first release containing core functionalities. From now on, except for point 6. below, each feature is a single module, so it can be released step by step.
	\item[5. Settings] to satisfy this functionality, the implementation of Account Manager ends. The User has the possibility to insert preferences and to set constraints. From now on, Solution Calculator processes travel solution taking into account the settings that the user sets.
	\item[6. Notification system] in this phase of the process, Notification Manager and his subcomponents (Notification Scheduler, Notification Dispatcher) are implemented. Note that, at this point, Notification Manager's utilizers (Incoming Appointment Scheduler, Weather and Traffic modules, Invitation Manager) are not present, so there are no actual notifications to send.
	\item[7. Incoming Appointments] to satisfy this functionality, the implementation, analysis, and test of Appointment Manager end after adding its last subcomponent, Incoming Appointment Scheduler. Now Notification Manager also notifies incoming appointments.
	\item[8. Wheather \& traffic updates] to satisfy this functionality, Weather and Traffic modules are implemented, analyzed and tested completely. Now Solution Calculator processes travel solutions taking into account bad weather and traffic. The user receives related notifications.
	\item[9. Invitations] to satisfy this functionality, Invitation Manager and related subcomponents are implemented, analyzed and tested. Now the user receives notifications concerning invitations.
	\item[10. Ticket purchase and management] to satisfy this functionality, Ticket Manager and related subcomponents are implemented, analyzed and tested.
	\item[11. Rent a sharing vehicle] to satisfy this functionality, part of Vehicle Sharing Manager is implemented, analyzed and tested.
	\item[12. Localization sharing vehicle] to satisfy this functionality, implementation, analysis and test of Vehicle Sharing Manager ends.\newline		
\end{description} \bigskip
\noindent
During integration test process (and before its termination), the following must subsist:
\begin{itemize}
	\item DB must be activated, it must contain an adequate quantity of data in order to test system's performances by overloading it. In this way, it is possible to test a big part of the components interfaced with the DBMS.
	\item Interactions with external actors must be present, both on the contractual and on the communicative level. In particular, at least two Transportation Companies, at least one Sharing Vehicle Company must interact with the system. This must subsist before Ticket Purchase and Management functionality will be completely implemented.  In this way, after their implementation, Ticket Manager and related subcomponents can be tested (with more attention on Purchase Manager) as soon as the implementation will be terminated. 
	Google must also be interfaced with the system before completing the implementation of Weather and Traffic update functionality. In this way we can test Weather and Traffic modules. 
	\item Payment service must be activated and it has to concern affiliated Transportation Companies. In this way, there is the possibility to test Ticket Manager and its subcomponents (e.g. Payment Handler). \newline	
\end{itemize} 
\bigskip 
We chose to implement, analyze and test at first, in a parallel way, Server and Mobile App. The Web App development will start as soon as the implementation and the test of all high and medium-high priority (for the user) features will be completed.