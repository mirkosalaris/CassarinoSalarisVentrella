\subsection{Weather and Traffic modules - Dynamic configuration}
	\label{sect:WeatherTrafficAlgorithm}
	As we \hyperref[sect:WeatherTrafficModules]{previously described} the Weather and Traffic modules are subject to load balancing and \defined{dynamic configuration} by the System Administrator. This two mechanism causes 4 events to happen. How they are managed is described below:
	\begin{description}[before={\renewcommand{\makelabel}[1]{-- \textit{##1}:}}]
		\item[a Notifier is deleted] this is the only activity that is not managed by the Manager because there is no reason for doing it. If the Notifier is simply being closed, it detaches itself from the Querier. If the Notifier crashes or it is suddenly closed, the Querier will notice this when trying to notify it and it will detach the dead Notifier. No other actions are needed.
		\item[a Notifier is created] the new Notifier communicates its zone to the Manager. The Manager will return to the Notifier a reference to the appropriate Querier using the Address Solver to interpret the zone of the Notifier. Thus, the Notifier can subscribe itself to the Querier.
		\item[Quierier is deleted] all the Notifiers previously attached to the Querier have to be analyzed. If a less specific Querier exists \textit{(see below)}, they are attached to it, otherwise they are put in a standby list (they will not receive any information about weather, or traffic)
		\item[a new Querier is instantiated] the standby list is scanned searching for Notifier that can be attached to the Querier (matching the two zone through the Address Solver). If a less specific Querier exists \textit{(see below)}, all the Notifier subscribed to it are analyzed and are eventually moved to the new Querier.
	\end{description}
	\medskip
	\textbf{Meaning of specificity of a Querier}\newline
	Let's assume, for instance, that there are four Querier: \inlinecode{ItalyQ}, \inlinecode{MilanQ}, \inlinecode{LazioQ}, \inlinecode{ParisQ}, respectively related to zones: Italy, Milan, Lazio, Paris. \inlinecode{MilanQ} and \inlinecode{LazioQ} are more specific with respect to \inlinecode{ItalyQ} because Milan and Lazio are inside the region Italy. On the other hand, \inlinecode{ItalyQ} is less specific with respect to \inlinecode{MilanQ} and \inlinecode{LazioQ}. \inlinecode{ParisQ} has no relation of specificity with all the others.\newline
	Hence:
	\begin{itemize}[label=--]
		\item \inlinecode{LazioQ} crashes $\rightarrow$ all the Notifiers subscribed to \inlinecode{LazioQ} are now moved to \inlinecode{ItalyQ}
		\item \inlinecode{RomeQ} Querier is created $\rightarrow$ \inlinecode{ItalyQ} is less specific than \inlinecode{RomeQ}, so \inlinecode{ItalyQ} is scanned searching for Notifiers associated to region Rome: if there are such Notifiers, they are moved to \inlinecode{RomeQ}. Notice: the Notifiers that were associated to \inlinecode{LazioQ} but are actually outside the region of Rome would remain associated to \inlinecode{ItalyQ}.
		\item \inlinecode{ParisQ} crashes $\rightarrow$ all the Notifiers associated to \inlinecode{ParisQ} are moved to the standby list, because no 'less specific' Queriers are available.
	\end{itemize}
	\smallskip
	Notice that Queriers are meant to be wide regions and not single cities, this was just an example.

\subsection{Solution Calculator - How it works}
		Here it is illustrated the procedure by which the S2B provides the user with the travel solutions for the selected appointment.
		As we \hyperref[sect:Select TravelPlan Solution]{previously described} the Solution Calculator takes care of handling the three-step  articulated process: recovery of feasible travelPlanSolutions, filtering and labeling.
	
	\paragraph{Recovery of feasible travelPlanSolutions:}
	\begin{itemize}
		\item  Solution Calculator, through Appointment Manager, retrieves appointment information (departure location, destination, time, presence of passengers, baggage, ..); 
		\item based on this information, Solution Calculator interfaces with Google, which provides it with a list of possible travel solutions - travelPlanSolutions  \textit{(each TravelPlan solution consists of more Rides each with its own TravelMean, distance, time)}.
	\end{itemize}

	\paragraph{Filtering}
	\begin{itemize}
		\item Solution Calculator, through Account Manager, retrieves user's TravelMeanConstraints and Eco-Friendly preference (each TravelMean constraint refers to a particular Travel Mean specifying the distance beyond which it is not to be used or a time band to be avoided);

		\item Solution Calculator, through Weather Manager and Traffic Manager, retrieves information about weather and traffic conditions (this takes into account the reliability of this information so they are only considered if the appointment is an \defined{Incoming Appointment});
		
		\item  for each TravelPlan in travelPlanSolutions, Solution Calculutor checks if there is a TravelPlan that does not meet the requirements, in particular: 
		
		\begin{itemize}	
			\item if the user has specified the presence of passengers and the TravelMean of the Ride is not authorized for the carriage of passengers, TravelPlan is removed by travelPlanSolutions;
			
			\item if the user has specified baggage and the Ride's TravelMean is not suitable for baggage's trasportation, TravelPlan is removed from travelPlanSolutions;
			
			\item if Ride violates a TravelMeanConstraint (for example, the Ride has a TravelMean for which there is a TravelMeanConstraint that says TravelMean should not be used for a distance > x and Ride has a distance > x), TravelPlan is removed by travelPlanSolutions. 
		\end{itemize}
		
		\end{itemize}
	
	
	
	\paragraph{Labeling}
		\mbox{}\newline 
		After filtering solutions in travelPlanSolutions, Solution Calculator:\newline
	\begin{itemize}

		
	
		\item calculates the cheapest  TravelPlan: the one for which the sum of all the estimated prices of individual Rides in TravelPlan results to be lowest if compared with the other TravelPlan in travelPlanSolutions  and sets a tag = "cheapest";
	
		\item calculates the fastest TravelPlan: the one for which the sum of all estimated Ride times of each Ride is the lowest if compared with the other TravelPlan in travelPlanSolutions and sets a tag = "fastest"; 
		
		\item calculates the most ecological TravelPlan: the one for which  the sum of all EcoScores of each Ride is  the highest if compared with the other TravelPlan in travelPlanSolutions and sets a tag = "Eco-Friendlest" (for this operation a function calculateEcoScore (TravelPlan) , detailed below, is used);
	
		\item for each TravelPlan where there is a Ride with a TravelMean is not recommended in case of adverse weather conditions (eg rain, snow, ..), if weather forecasts are bad, set a tag = "warning: Weather";\newline
	
		\item for each TravelPlan where there is a Ride crossing an area where there is intense traffic, set a tag="warning: traffic". 
		\end{itemize}
	
		Finally Solution Calculator sends TravelPlanSolutions to the user.\newline
	
	\filbreak
	\paragraph{Calculate EcoScore:}	
	\mbox{} \newline
		Each TravelMean has associated a travelMeanEcoPenalty, more the TravelMean is 'Eco-Friendly', less the penalty is. 'The algorithm attributes to each TravelPlan an EcoScore considering all the Rides that belong to it.
		The maximum possible EcoScore is 0; \newline
	
	PseudoCode: \newline
	\begin{lstlisting}[language=python, numbers=left]
	calculateEcoScore(TravelPlan tp){
		score = 0;
		for each Ride r in tp{
			score = score - r.travelMean.penalty * r.distance;
		}
		tp.Ecoscore = score;
	}
	\end{lstlisting}